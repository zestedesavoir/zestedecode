\documentclass{beamer}

\usepackage[utf8]{inputenc}

\title{Objectifs de l'atelier}
\author{Zeste de Savoir}
\usetheme{zestedesavoir}
\begin{document}

\begin{frame}
  \titlepage
\end{frame}

\begin{frame}
    \frametitle{Objectif 1}
        \begin{itemize}
            \item Afficher l'image de fond dans la fenêtre du jeu
            \item Afficher les murs de cactus au bord du terrain
        \end{itemize}
\end{frame}

\begin{frame}
    \frametitle{Objectif 2}
        \begin{itemize}
            \item Afficher le corps du serpent à l'écran
            \item Lui ajouter la tête et la queue
        \end{itemize}
\end{frame}

\begin{frame}
    \frametitle{Objectif 3}
        \begin{itemize}
            \item Faire en sorte que le serpent avance quand on appui sur la bonne touche
            \item Faire tourner le serpent à droite et à gauche avec les autres touches
            \item Veiller à placer les morceaux du corps dans le bon sens !
        \end{itemize}
\end{frame}

\begin{frame}
    \frametitle{Objectif 4}
        \begin{itemize}
            \item Faire apparaître des pommes sur le terrain
            \item Permettre au serpent de manger la pomme
            \item Ne pas faire apparaître de pommes sur les cactus !
        \end{itemize}
\end{frame}

\begin{frame}
    \frametitle{Objectif 5}
        \begin{itemize}
            \item Fin du jeu si le serpent touche un cactus...
            \item ... ou bien si le serpent se mord lui-même
        \end{itemize}
\end{frame}

\end{document}