\documentclass{beamer}

\usepackage[utf8]{inputenc}
\usepackage[ddmmyyyy]{datetime}

\title{Métiers et études de l'informatique}
\author{Zeste de Savoir}
\usetheme{zestedesavoir}
\begin{document}

\begin{frame}
  \titlepage
\end{frame}

\begin{frame}
    \frametitle{Métiers de l'informatique}
    Métiers en rapport avec le développement :
        \begin{itemize}
            \item Chef de projet
            \item Développeur (regroupe en fait des postes très différents selon les situations)
            \item Architecte logiciel
            \item Contrôle qualité (tests...)
            \item ...
        \end{itemize}
\end{frame}

\begin{frame}
    \frametitle{Métiers de l'informatique}
    Le développement touche à des secteurs très différents :
        \begin{itemize}
            \item Logiciels professionnels
            \item Logiciels grand public
            \item Logiciels spécialisés : antivirus...
            \item Jeu vidéo
            \item Informatique embarqué : voitures, montres, satellites...
            \item Sites internet, applications mobiles, tablette, télévision, consoles de jeu...
        \end{itemize}
\end{frame}

\begin{frame}
    \frametitle{Métiers de l'informatique}
    Métiers en rapport avec le développement web :
        \begin{itemize}
            \item Chef de projet
            \item Développeur web (front, back ...)
            \item UI, UX designer
            \item Expert en référencement
            \item Autres non liés à l'info : graphiste, chargé de communication, web-marketing...
        \end{itemize}
\end{frame}

\begin{frame}
    \frametitle{Métiers de l'informatique}
    Métiers en rapport avec les systèmes et réseaux :
        \begin{itemize}
            \item Administrateur réseaux
            \item Architecte réseaux
            \item Administrateur système
            \item Support technique (Helpdesk)
        \end{itemize}
\end{frame}

\begin{frame}
    \frametitle{Métiers de l'informatique}
    Métiers en rapport avec les données :
        \begin{itemize}
            \item Administrateur de bases de données
            \item Data scientist, data analyst, data miner...
        \end{itemize}
\end{frame}

\begin{frame}
    \frametitle{Métiers de l'informatique}
    Métiers en rapport avec l'enseignement et la recherche :
        \begin{itemize}
            \item Enseignant chercheur en informatique
            \item Chercheur en informatique
        \end{itemize}
    
    Dans le public ou dans le privé.
\end{frame}

\begin{frame}
    \frametitle{Études de l'informatique}
    Cursus courts :
        \begin{itemize}
            \item BTS (bac+2) : SIO (options SLAM et SISR) ou SN (option Informatique et Réseaux)
            \item DUT (bac+2) : Info, MMI, GEII
            \item Licence pro (bac+3) : Beaucoup de spécialités, à faire après un BTS ou un DUT
            \item Formations courtes privées
        \end{itemize}
\end{frame}

\begin{frame}
    \frametitle{Études de l'informatique}
    Cursus plus longs :
        \begin{itemize}
            \item Cursus licence master à l'université (bac+3 puis bac+5)
            \item Écoles d'ingénieur (bac+5), généraliste ou spécialisée\\
            (directement après le bac, après une CPGE ou après un bac+2/3)
            \item Écoles privées : beaucoup d'écoles différentes
        \end{itemize}
        \medbreak
        Le doctorat est une formation à la recherche scientifique de 3 ans qui s'effectue après un master ou une école d'ingénieur.
\end{frame}

\end{document}